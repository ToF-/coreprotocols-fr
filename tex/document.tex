\documentclass{book}

\usepackage[utf8]{inputenc}
\usepackage[french]{babel}
\usepackage[T1]{fontenc}
\usepackage{lmodern}

\usepackage{nameref}
\newcommand*{\numref}[1]{{\hyperref[{#1}]{\autoref*{#1}}}}
\newcommand*{\fullref}[1]{\textit{\hyperref[{#1}]{\autoref*{#1} : \nameref*{#1}}}}

\usepackage[size=novel]{createspace}

\author{Richard Kasperowski}
\title{Les Protocoles Fondamentaux}

\pdftitle{Les Protocoles Fondamentaux}
\pdfauthor{Richard Kasperowski}
\pdfsubject{Les Core Protocols}

\begin{document}

\tableofcontents

\chapter{Comment utiliser ce livre ?} \label{utiliser-ce-livre}

Ce livre a pour vocation de partager plus largement les Protocoles Fondamentaux. Il consiste en un résumé succinct des 
Protocoles Fondamentaux dans un livre à prix raisonnable et un libre électronique. Vous pouvez lire ce guide en tant que 
tel ou en conjonction avec l'excellent livre de Jim et Michele McCarthy, \emph{Software for your head}, paru en 2001
chez Addison-Wesley et dont ce document est inspiré. Pour les Protocoles Fondamentaux, lisez \fullref{engagements} et 
\fullref{core-protocols}. 

Ce livre est également un guide pour faciliter l'animation d'ateliers. Ces ateliers visent à vous aider, ainsi que 
votre équipe, á atteindre l'excellence. Un atelier peut durer de quelques heures à quelques jours ; de longs ateliers
peuvent permettre de meilleurs résultats. L'atelier est explicité dans le protocole additionnel, \nameref{atelier-express}
dans le \numref{aller-plus-loin}. Ce chapitre inclut également des éléments issus du \emph{BootCamp Manual V2.3} édité en 
2001 chez McCarthy Technologies, Inc. et de correspondances personnelles par courrier électronique avec Jim McCarthy. Pour 
préparer un atelier, ou pour en apprendre plus à propos des Protocoles Fondamentaux, lisez ce livre intégralement. 

Pour profiter pleinement et intégrer les Protocoles Fondamentaux, participez à un \emph{Core Procols BootCamp}
\footnote{Formation aux Protocoles Fondamentaux (ndt)}. Cette formation consiste en une session de quelques jours qui aidera
votre équipe à concevoir, à mettre en place et à livrer d'excellents produits à temps. Pour organiser ou participer à cette
formation, contactez un facilitateur de \emph{Core Protocols BootCamp}.

\chapter{Les Engagements Fondamentaux} \label{engagements}

Les Engagements Fondamentaux (\emph{Core Commitments}) ont été définis par Jim McCarthy et Michele McCarty. 

\begin{enumerate}
	\item Je m'engage lorsque je suis présent
	\begin{enumerate}
		\item à avoir conscience de et à divulguer
		\begin{enumerate}		
			\item ce que je veux
			\item ce que je pense
			\item ce que je ressens
		\end{enumerate}
		\item à toujours chercher efficacement de l'aide 
		\item à m'abstenir de donner et à refuser d'accepter toute transmission émotionnelle incohérente
		\item à immédiatement, lorsque j'ai ou lorsque j'entends une idée plus pertinente que celle qui domine 
		      à ce moment, 
		\begin{enumerate}		
			\item la soumettre pour qu'elle soit acceptée ou refusée
			\item explicitement chercher à l'améliorer
		\end{enumerate}
		\item à personnellement soutenir la meilleure idée
		\begin{enumerate}		
			\item d'où qu'elle vienne
			\item quelle que soit mon espérance de voir une meilleure idée émerger
			\item quand je n'ai pas de meilleure alternative à proposer
		\end{enumerate}
	\end{enumerate}
	\item Je m'engage à chercher à percevoir plus que je ne cherche à être perçu
	\item Je m'engage à tirer parti des équipes, tout particulièrement pour venir à bout de tâches difficiles
	\item Je m'engage à parler quand et seulement quand je pense pouvoir améliorer le rapport résultat / efforts courant
	\item Je m'engage à ne proposer et à n'accepter que des comportements et des échanges raisonnables et axés vers le résultat
	\item Je m'engage à quitter les situations moins constructives
	\begin{enumerate}
		\item quand je ne peux pas tenir les engagements demandés
		\item quand je peux prendre part à quelque chose de plus important
	\end{enumerate}
	\item Je m'engage à faire maintenant ce qui doit être fait et qui peut effectivement être fait maintenant
	\item Je m'engage à avancer vers un but donné et à adapter mon comportement pour l'atteindre
	\item Je m'engage à utiliser les Protocoles Fondamentaux (ou mieux) si possible
	\begin{enumerate}
		\item j'utiliserai de façon appropriée et sans porter préjudice les Contrôles de Protocole
	\end{enumerate}
	\item Je m'engage à ne jamais blesser -- ni à tolérer que quelqu'un blesse -- qui que ce soit pour sa fidélité à ces engagements
	\item Je m'engage à ne jamais rien faire d'idiot à dessein
\end{enumerate}

\chapter{Les Protocoles Fondamentaux} \label{core-protocols}

\section{Passer (\emph{Pass})}
\subsection{Étapes}
\subsection{Engagements}
\subsection{Remarques}

\section{Initialisation (\emph{Check In})}
\subsection{Étapes}
\subsection{Engagements}
\subsection{Remarques}

\section{Terminaison (\emph{Check Out})}
\subsection{Étapes}
\subsection{Engagements}
\subsection{Remarques}

\section{Demande d'aide (\emph{Ask for help})}
\subsection{Étapes}
\subsection{Engagements}
\subsection{Remarques}

\section{Contrôle de protocole (\emph{Protocol check})}
\subsection{Étapes}
\subsection{Engagements}
\subsection{Remarques}

\section{Contrôle d'intention (\emph{Intention check})}
\subsection{Étapes}
\subsection{Engagements}
\subsection{Remarques}

\section{Décideur (\emph{Decider})}
\subsection{Étapes}
\subsection{Engagements}
\subsection{Remarques}

\section{Résolution (\emph{Resolution})}
\subsection{Étapes}
\subsection{Engagements}
\subsection{Remarques}

\section{Jeu de la perfection (\emph{Perfection game})}
\subsection{Étapes}
\subsection{Engagements}
\subsection{Remarques}

\section{Alignement personnel (\emph{Personal alignment})}
\subsection{Étapes}
\subsection{Engagements}
\subsection{Remarques}

\section{Enquêter (\emph{Investigate})}
\subsection{Étapes}
\subsection{Engagements}
\subsection{Remarques}

\chapter{Aller plus loin, protocoles} \label{aller-plus-loin}

\section{Atelier express} \label{atelier-express}
\section{Engagements personnels} \label{engagements-personnels}
\section{Alignement personnel express} \label{alignement-personnel-express}
\section{Toile d'alignement express} \label{toile-alignement-express}
\section{Vision partagée} \label{vision-partagee}

\end{document}
