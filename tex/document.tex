\documentclass{book}

\usepackage[utf8]{inputenc}
\usepackage[french]{babel}
\usepackage[T1]{fontenc}
\usepackage{lmodern}

\usepackage{nameref}
\newcommand*{\numref}[1]{{\hyperref[{#1}]{\autoref*{#1}}}}
\newcommand*{\fullref}[1]{\textit{\hyperref[{#1}]{\autoref*{#1} : \nameref*{#1}}}}

\usepackage[size=novel]{createspace}

\author{Richard Kasperowski}
\title{Les Protocoles Fondamentaux}

\pdftitle{Les Protocoles Fondamentaux}
\pdfauthor{Richard Kasperowski}
\pdfsubject{Les Protocoles Fondamentaux}

\begin{document}

\frontmatter

\setcounter{secnumdepth}{1}
\setcounter{tocdepth}{1}
\tableofcontents

\setlength{\parskip}{0.5em}

\chapter{Avant-propos}

Plus j'étudie, pratique et partage les Protocoles Fondamentaux, plus je les considère comme la meilleure façon d'atteindre
d'excellents résultats dans toutes les facettes de ma vie et plus je me sens aligné avec la vision de Jim et Michele McCarthy :
que tous vivent dans l'excellence.

J'ai rassemblé les Protocoles Fondamentaux dans un livre autonome pour les rendre plus accessibles à ceux que j'aime, pour 
qu'elles·ils puissent mieux me comprendre, pour que nous puissions pratiquer le \emph{Core} ensemble et pour atteindre
l'excellence ensemble.

Merci, Jim et Michele McCarthy ainsi que tous ceux qui ont contribué aux Protocoles Fondamentaux. Vous m'aidez à atteindre
l'excellence.

--- Richard Kasperowski


\chapter{Licence}

Les Protocoles Fondamentaux (\emph{The Core Protocols}) V.3.03

Copyright \copyright{} Jim McCarthy et Michele McCarthy

\emph{The Core} est distribué selon les termes de la GNU General Public Licence telle que publiée par la Free Software 
Foundation, soit dans sa version 3 ou (à votre convenance) dans toute version ultérieure. Pour le contenu exact, voir
\url{http://www.gnu.org/licenses/}. \emph{The Core} est considéré comme un code source en vertu de cet accord. Chacun est autorisé
à copier et distribuer des copies conformes ou modifiées de ce contenu à condition que vous distribuiez également celui-ci
dans son intégralité en incluant ce paragraphe. 

Les Protocoles Fondamentaux: un guide vers l'excellence, basé sur les travaux de Jim McCarthy et Michele McCarthy 
(\emph{The Core Protocols: A Guide to Greatness, based on the work of Jim McCarthy and Michele McCarthy})

Copyright \copyright{} Richard Kasperowski

Ce livre est distribué selon les termes de la GNU General Public Licence telle que publiée par la Free Software 
Foundation, soit dans sa version 3 ou (à votre convenance) dans toute version ultérieure. Pour le contenu exact, voir
\url{http://www.gnu.org/licenses/}. Ce livre est considéré comme un code source en vertu de cet accord. Chacun est autorisé
à copier et distribuer des copies conformes ou modifiées de ce contenu à condition que vous distribuiez également celui-ci
dans son intégralité en incluant ce paragraphe. 

\mainmatter

\chapter{Comment utiliser ce livre ?} \label{utiliser-ce-livre}

Ce livre a pour vocation de partager plus largement les Protocoles Fondamentaux. Il consiste en un résumé succinct des 
Protocoles Fondamentaux dans un livre à prix raisonnable et un libre électronique. Vous pouvez lire ce guide en tant que 
tel ou en conjonction avec l'excellent livre de Jim et Michele McCarthy, \emph{Software for your head}, paru en 2001
chez Addison-Wesley et dont ce document est inspiré. Pour les Protocoles Fondamentaux, lisez \fullref{engagements} et 
\fullref{core-protocols}. 

Ce livre est également un guide pour faciliter l'animation d'ateliers. Ces ateliers visent à vous aider, ainsi que 
votre équipe, á atteindre l'excellence. Un atelier peut durer de quelques heures à quelques jours ; de longs ateliers
peuvent permettre de meilleurs résultats. L'atelier est explicité dans le protocole additionnel, \nameref{atelier-express}
dans le \numref{aller-plus-loin}. Ce chapitre inclut également des éléments issus du \emph{BootCamp Manual V2.3} édité en 
2001 chez McCarthy Technologies, Inc. et de correspondances personnelles par courrier électronique avec Jim McCarthy. Pour 
préparer un atelier, ou pour en apprendre plus à propos des Protocoles Fondamentaux, lisez ce livre intégralement. 

Pour profiter pleinement et intégrer les Protocoles Fondamentaux, participez à un \emph{Core Procols BootCamp}
\footnote{Formation aux Protocoles Fondamentaux (ndt)}. Cette formation consiste en une session de quelques jours qui aidera
votre équipe à concevoir, à mettre en place et à livrer d'excellents produits à temps. Pour organiser ou participer à cette
formation, contactez un facilitateur de \emph{Core Protocols BootCamp}.

\chapter{Les Engagements Fondamentaux} \label{engagements}

Les Engagements Fondamentaux (\emph{Core Commitments}) ont été définis par Jim McCarthy et Michele McCarty. 

\begin{enumerate}
	\item Je m'engage lorsque je suis présent
	\begin{enumerate}
		\item à avoir conscience de et à divulguer
		\begin{enumerate}		
			\item ce que je veux
			\item ce que je pense
			\item ce que je ressens
		\end{enumerate}
		\item à toujours chercher efficacement de l'aide 
		\item à m'abstenir de donner et à refuser d'accepter toute transmission émotionnelle incohérente
		\item à immédiatement, lorsque j'ai ou lorsque j'entends une idée plus pertinente que celle qui domine 
		      à ce moment, 
		\begin{enumerate}		
			\item la soumettre pour qu'elle soit acceptée ou refusée
			\item explicitement chercher à l'améliorer
		\end{enumerate}
		\item à personnellement soutenir la meilleure idée
		\begin{enumerate}		
			\item d'où qu'elle vienne
			\item quelle que soit mon espérance de voir une meilleure idée émerger
			\item quand je n'ai pas de meilleure alternative à proposer
		\end{enumerate}
	\end{enumerate}
	\item Je m'engage à chercher à percevoir plus que je ne cherche à être perçu
	\item Je m'engage à tirer parti des équipes, tout particulièrement pour venir à bout de tâches difficiles
	\item Je m'engage à parler quand et seulement quand je pense pouvoir améliorer le rapport résultat / efforts courant
	\item Je m'engage à ne proposer et à n'accepter que des comportements et des échanges raisonnables et axés vers le résultat
	\item Je m'engage à quitter les situations moins constructives
	\begin{enumerate}
		\item quand je ne peux pas tenir les engagements demandés
		\item quand je peux prendre part à quelque chose de plus important
	\end{enumerate}
	\item Je m'engage à faire maintenant ce qui doit être fait et qui peut effectivement être fait maintenant
	\item Je m'engage à avancer vers un but donné et à adapter mon comportement pour l'atteindre
	\item Je m'engage à utiliser les Protocoles Fondamentaux (ou mieux) si possible
	\begin{enumerate}
		\item j'utiliserai de façon appropriée et sans porter préjudice les Contrôles de Protocole
	\end{enumerate}
	\item Je m'engage à ne jamais blesser -- ni à tolérer que quelqu'un blesse -- qui que ce soit pour sa fidélité à ces engagements
	\item Je m'engage à ne jamais rien faire d'idiot à dessein
\end{enumerate}

\chapter{Les Protocoles Fondamentaux} \label{core-protocols}

\section{Passer (\emph{Pass})} \label{protocole-passer}

\emph{Passer} est un moyen de décliner sa participation à une activité. Utilisez-le chaque fois que vous ne souhaitez pas 
participer à une activité.

\subsection{Étapes}
\begin{enumerate}
	\item Quand vous avez décidé de ne pas participer à une activité, dites \og{}je passe\fg{}.
	\item \og{}Dé-passez\fg{} dès que vous le souhaitez. Reprenez l'activité dès que vous savez que vous voulez participer ; 
	      dites alors \og{}je participe\fg{}.
\end{enumerate}

\subsection{Engagements}
\begin{itemize}
	\item Gardez pour vous les raisons qui vont ont fait passer.
	\item Annoncez votre intention de passer dès lors que vous avez pris votre décision.
	\item Respectez le droit de chacun de passer sans avoir à fournir d'explication.
	\item Soutenez ceux qui passent en ne remettant pas en question leur décision.
	\item Si quelqu'un passe, ne le jugez pas, ne l'humiliez pas, ne l'interrogez pas, ne le punissez pas. 
\end{itemize}

\subsection{Remarques}
\begin{itemize}
	\item En général, vous ne serez pas aligné avec vos Engagements Fondamentaux si vous passez la plupart du temps.
	\item Vous pouvez choisir de passer sur n'importe quelle activité. Cependant, si vous avez adopté les Engagements
	      Fondamentaux, vous ne pouvez pas passer sur un vote \emph{Décideur} et vous devez dire \og{}je suis présent\fg{}
	      durant une initialisation.
	\item Vous pouvez décider de passer même si vous avez commencé l'activité.
\end{itemize}

\section{Initialisation (\emph{Check In})} \label{initialisation}

Commencez une réunion par une \emph{Initialisation}. Vous pouvez également en faire une en tête à tête ou en groupe chaque fois que 
cela améliorerait les interactions d'une équipe.

\subsection{Étapes}
\begin{enumerate}
	\item Celui qui a la parole dit : \og{}Je ressens \emph{telle émotion}\fg{}. Cela peut être de la tristesse, de la colère, de la joie, de la peur ;
	      il peut en citer plusieurs. 
	      \begin{itemize}
	      	\item Il peut, s'il le souhaite, ajouter une brève explication.
	      	\item Ou, si d'autres se sont déjà exprimé lors de l'initialisation, il peut dire \og{}je passe\fg{} (voir \fullref{protocole-passer}). 
	      \end{itemize}
	\item Celui qui a la parole dit \og{}je suis présent\fg{}. Cela signifie qu'il prend acte des Engagements Fondamentaux. 
	\item Le groupe répond \og{}bienvenue\fg{}. 
\end{enumerate}

\subsection{Engagements}
\begin{itemize}
	\item Exprimez vos émotions sans les qualifier.
	\item N'exprimez que vos émotions (ne parlez par pour les autres).
	\item N'interrompez pas l'initialisation de quelqu'un d'autre : gardez le silence.
	\item Ne faites pas référence à l'initialisation de quelqu'un d'autre sans avoir eu son accord explicite.
\end{itemize}

\subsection{Remarques}
\begin{itemize}
	\item Dans le cadre des Protocoles Fondamentaux, toutes les émotions sont exprimées comme des combinaisons de tristesse, de colère,
	      de joie et de peur.
	      \begin{itemize}
	      	\item La joie reflète un gain.
	      	\item La tristesse reflète une perte.
	      	\item La colère reflète un problème.
	      	\item La peur reflète l'inconnu.
	      \end{itemize}
	      Par exemple, l'excitation peut-être exprimée comme une combinaison de joie et de peur.
	\item Soyez aussi profond que possible dans l'initialisation. La plupart du temps, vous citerez une ou deux émotions. 
	\item N'essayez pas de minimiser vos émotions. Ne dites pas que vous être \og{}un peu\fg{} en colère ; en revanche, vous pouvez nuancer : 
	      \og{}je suis en colère mais je ressens aussi de la joie\fg{}.
	\item À moins que le groupe ne soit très grand, si plus d'une personne participe à l'initialisation, il est préférable que tous le groupe
	      participe également.
	\item \emph{Heureux} peut se traduire par de la joie. \emph{Inquiet} peut se traduire par de la peur.
\end{itemize}

\section{Terminaison (\emph{Check Out})}

Votre présente physique indique que vous respectez les Engagements Fondamentaux. Lorsque vous prenez conscience que vous ne pouvez plus
respecter ces engagements ou que vous pensez être plus utile ailleurs, vous devez annoncer la \emph{Terminaison}.

\subsection{Étapes}
\begin{enumerate}
	\item Dites \og{}je sors\fg{}.
	\item Quittez physiquement le groupe jusqu'à ce que vous soyez prêt à rejoindre le groupe à nouveau avec une \emph{Initialisation} (\numref{initialisation}).
	\item Si vous savez lorsque vous allez revenir et que l'information est utile, vous pouvez éventuellement en informer le groupe.
	\item Les membres du groupe présents au moment de la terminaison ne doivent pas suivre la personne partante. Ils ne doivent pas non plus lui parler ou 
	      parler de lui.
\end{enumerate}

\subsection{Engagements}
\begin{itemize}
	\item Revenez dès que vous êtes en mesure de respecter les Engagements Fondamentaux.
	\item Revenez avec une \emph{Initialisation} sans plus attirer l'attention du groupe sur vous.
	\item Si quelqu'un sort, ne le jugez pas, ne l'humiliez pas, ne l'interrogez pas, ne le punissez pas. 
\end{itemize}

\subsection{Remarques}
\begin{itemize}
	\item Lorsque vous sortez, faites-le aussi calmement et discrètement que vous pouvez pour déranger le moins possible les autres.
	\item Sortez si votre état émotionnel entrave votre capacité à avancer, si vous n'êtes pas réceptif à de nouvelles informations ou 
	      si vous ne savez pas ce que vous voulez.
	\item La terminaison acte votre incapacité à contribuer à cet instant.
\end{itemize}

\section{Demande d'aide (\emph{Ask for help})}

La \emph{Demande d'aide} vous permet de bénéficier efficacement des connaissance et des compétences des autres. Demander de l'aide
catalyse le lien et la vision partagée. Utilisez-la toujours, que ce soit avant ou pendant la poursuite d'un objectif. 

\subsection{Étapes}
\begin{enumerate}
	\item Le demandeur demande à un autre : \og{}[\textsc{destinataire}], peux-tu [\textsc{demande}] ?\fg{}.
	\item Le demandeur précise les détails ou les contraintes associées à sa demande.
	\item Le destinataire de la demande répond avec \og{}oui\fg{}, \og{}non\fg{} ou en proposant une aide alternative.
\end{enumerate}

\subsection{Engagements}
\begin{itemize}
	\item Commencez toujours votre demande par \og{}peux-tu\fg{}.
	\item Ayez une vision claire de ce dont vous avez besoin ou, si ce n'est pas le cas, indiquez-le en disant : \og{}Je ne suis pas sûr 
	      de ce dont j'ai besoin, mais peux-tu m'aider ?\fg{}.
    \item Partez du principe que les porteurs d'aide sont toujours disponibles et qu'il est de leur seule responsabilité de dire \og{}non\fg{}.
    \item Dites \og{}non\fg{} si vous ne voulez pas apporter votre aide.
    \item Acceptez le \og{}non\fg{} sans poser de question et sans le prendre personnellement.
    \item Soyez réceptif à l'aide que l'on vous propose.
    \item Aidez du mieux que vous pouvez, même si ce n'est pas ce que le demandeur a exprimé.
    \item Reportez votre demande d'aide si vous n'êtes pas en mesure de vous y consacrer pleinement.
    \item Demandez des précisions si vous ne comprenez pas tous les détails d'une demande d'aide.
    \item Ne vous excusez pas de demander de l'aide.
\end{itemize}

\subsection{Remarques}
\begin{itemize}
	\item Demander de l'aide ne coûte rien. Au pire, un \og{}non\fg{} ne vous avancera pas ; au mieux, vous avez gagné du temps pour
	      accomplir ou apprendre quelque chose.
	\item Les donneurs d'aide doivent répondre \og{}non\fg{} s'ils ne sont pas certains de vouloir aider ; une fois la demande déclinée,
	      ils n'ont rien à ajouter.
	\item Vous ne demandez jamais \og{}trop\fg{} d'aide tant que l'on ne vous l'a pas signifié.
	\item Si vous doutez de la pertinence de l'aide qui vous est offerte ou si vous ne l'estimez pas utile, ou encore si vous considérez 
	      que vous avez évalué et rejeté l'idée qui vous est proposée, adoptez une attitude de curiosité au lieu d'une opposition automatique
	      \og{}mais...\fg{}.
	\item Demander de l'aide lorsque vous êtes en difficulté montre que vous avez attendu trop longtemps. Demander de l'aide quand tout va bien.
	\item Échangez simplement avec quelqu'un, même si elle·il ne sait rien de votre problème. Elle·il peut vous aider à faire émerger des idées,
	      particulièrement si la demande est de vous faire \emph{Enquêter}.
\end{itemize}

\section{Contrôle de protocole (\emph{Protocol check})}

Utilisez le \emph{Contrôle de protocole} si vous pensez qu'un protocole est mal utilisé ou lorsqu'un Engagement Fondamental n'est pas
respecté.

\subsection{Étapes}
\begin{enumerate}
	\item Dites \og{}contrôle de protocole\fg{}.
	\item Si vous êtes en mesure d'expliquer l'usage correct du protocole, faites-le. Sinon, demandez de l'aide.
\end{enumerate}

\subsection{Engagements}
\begin{itemize}
	\item Dites \og{}contrôle de protocole\fg{} dès que vous prenez conscience d'un usage incorrect d'un protocole ou du non-respect
	      d'un Engagement Fondamental. Faites-le indépendamment de ce que vous ou le groupe étiez en train de faire à ce moment. 
	\item Soutenez toute personne demandant un \emph{Contrôle de protocole}.
	\item Ne blâmez ni n'humiliez personne demandant un \emph{Contrôle de protocole}.
	\item Demandez de l'aide dès que vous prenez conscience que vous n'êtes pas certain de la façon correct d'utiliser un protocole.
\end{itemize}

\section{Contrôle d'intention (\emph{Intention check})}
\subsection{Étapes}
\subsection{Engagements}
\subsection{Remarques}

\section{Décideur (\emph{Decider})}
\subsection{Étapes}
\subsection{Engagements}
\subsection{Remarques}

\section{Résolution (\emph{Resolution})}
\subsection{Étapes}
\subsection{Engagements}
\subsection{Remarques}

\section{Jeu de la perfection (\emph{Perfection game})}
\subsection{Étapes}
\subsection{Engagements}
\subsection{Remarques}

\section{Alignement personnel (\emph{Personal alignment})}
\subsection{Étapes}
\subsection{Engagements}
\subsection{Remarques}

\section{Enquêter (\emph{Investigate})}
\subsection{Étapes}
\subsection{Engagements}
\subsection{Remarques}

\chapter{Aller plus loin, protocoles} \label{aller-plus-loin}

\section{Atelier express} \label{atelier-express}
\section{Engagements personnels} \label{engagements-personnels}
\section{Alignement personnel express} \label{alignement-personnel-express}
\section{Toile d'alignement express} \label{toile-alignement-express}
\section{Vision partagée} \label{vision-partagee}

\end{document}
